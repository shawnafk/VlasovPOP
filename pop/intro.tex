\section{Introduction}
\label{sec:intro}
Whistler-mode chorus is a special electromagnetic emission in whistler range of frequency, frequently observed in the planetary magnetosphere \cite{helliwell1965whistlers,burtis_magnetospheric_1976,tsurutani_postmidnight_1974}, and recently studied in the laboratory plasmas \cite{vancompernolle2015}.
The formation and evolution of chorus wave is related with the energy transference between wave and energetic electrons.
It plays a key role in particle acceleration processes in the radiation belt \cite{horne_wave_2005,thorne_rapid_2013,reeves_electron_2013} and pulsating and diffuse aurora in the atmosphere \cite{nishimura_identifying_2010,kasahara_pulsating_2018,thorne_scattering_2010}, which have received a significant research interest, such as in the studies of the superradiance in free-electron lasers \cite{zonca_nonlinear_2021, soto-chavez2012}.   

The physics of the chorus wave in essence are the nonlinear wave-particle interactions \cite{oneil1971,oneil1972} between the resonant trapped electrons and the whistler waves \cite{omura_theory_2008, an2019}. 
The similar nonlinear wave-particle interactions, such as the Alfv\'en wave instabilities \cite{chen2016,wang2018,wang2012,wang2012a} and energetic-particle-driven geodesic acoustic modes \cite{wang2013}, have been studies extensively in the fusion related plasma.
Such instabilities also involve the mode frequency sweeping and lead to premature ejection of alpha particles that deteriorate plasma confinement \cite{fasoli2007}.
The dynamics of nonlinear resonant particle can be modeled by Berk-Brizman (BB) model \cite{berk1990,berk1990a,berk1990b,berk1996,berk1997,berk1999}, which is a general theoretic model based on the bump-on-tail (BOT) paradigm. 
The original version and the advanced versions \cite{lilley2009,lilley2014a,vann2007,hezaveh2021,hezaveh2017,breizman2010,hezaveh2020} focus on the uniform regime, where the wave is treated as stationary with fixed wave number $k$. 
The spontaneous hole and clump structure and their evolution in the phase space were revealed and well explains the frequency chirping of the excited wave.
For the chorus wave case, the phase space behavior of the resonant electrons also plays an important role for the  chirping behavior.
It has been long time discussed \cite{sudan_theory_1971,vomvoridis_theory_1982,dysthe_studies_1971,nunn_self-consistent_1974,omura1991} and quantitatively explains the frequency chirping rate and nonlinear wave growth of chorus wave at various locations along the magnetic field \cite{tao_theoretical_2020,omura_nonlinear_2021,zonca_nonlinear_2021,zonca2022}.

However, although the BB model for the BOT paradigm was applied to study the chorus chirping problem \cite{soto-chavez2014}, a key feature for the chorus wave in the planetary magnetosphere is the involved inhomogeneous magnetic field. 
It was first noticed by Helliwell in his kinematic theory of the ``consistent-wave'' concept in the 1960s \cite{helliwell_theory_1967}.
It elucidates the change of chirping rate with respect to magnetic field inhomogeneity.
In more recent studies \cite{wu_controlling_2020,fujiwara2023}, it is found that the frequency chirping behavior can be controlled by changing the magnetic field configuration. Both downward chirping and bi-direction chirping were reproduced.
And a phenomenological chirping model called Trap-Release-Amplify (TaRA) model \cite{tao_trap-release-amplify_2021} was proposed and explained how frequency chirping occurs and why chirping direction with respect to the sign of magnetic field gradient.
Although the model has successfully explained chirping behavior in planetary magnetosphere, including that on Mars \cite{teng2023}, it raises the intriguing question of how rapidly varying chirping elements and resonant electrons are influenced by the background magnetic field's inhomogeneity.
This is despite the fact that the scale of the field's inhomogeneity is significantly larger than the characteristic scales of rapid changes and fine structure.
Besides, numerical simulations of artificially triggered emissions show different mechanism for the formation of chorus wave \cite{nogi2022,nogi2023}. 
The wave source moves with a velocity that is a combination of the group velocity and resonance velocity, rather than the velocity alone. This suggests that the chirping may not be emitted by the released particle.

Anyhow, in current theoretical and numerical studies, many questions still remain regarding the fine structures of the chorus observed in the magnetosphere. \cite{zhang2021}. 
The chirping mechanism with respect to the nonlinear wave particle interaction and the fine structures of the chorus wave are also addressed in the recently studies \cite{omura_nonlinear_2021,gao2014a,li2019b}.
Additional information about the particle dynamics and wave evolution, such as high resolution phase-space structure, fine wave number and frequency results, are still desired in additional to the satellite observations.
The numerical modelings are favorable for solving such issue \cite{katoh_computer_2007,vhscode,xiao2015,xiao2020explicit,xie2022,tao_numerical_2014,chen2015a}
However, due to introduce background magnetic field inhomogeneity, the simulation for the chorus wave is essentially multiscale.
Finding an appropriate method to separate the scales of resonant particles is a complex task because of the intricate interactions between waves and particles along the inhomogeneous magnetic field. 
Challenges and the need have spurred the development of a new theoretical and numerical model.

In this paper, we go over the dynamics of resonant electrons and the evolution of whistler wave in Earth inner magnetosphere.
The key feature in our theory is the scale separation of electron motion to the slowly one along the magnetic field and the fast wave interaction. 
To do so, we build a Hamiltonian theory in the reference frame moving with local resonance.
We divide the continuous spatial domain of the single magnetic field line into a group of cells representing a group of local reference.
By expanding the phase of the wave and the resonance electrons, a canonical Hamiltonian on the cell is constructed which naturally separated the fast and slowly varying motion of resonant electrons.
The canonical motion equations are then derived, which governs the dynamics of resonant electrons. 
The evolution of whistler-mode wave is determined by the Ampere equation, in which the wave is expressed in its eikonal form, thus, the frequency and wave number can be given directly and accurately.
With the established theory, we then discuss the onset stage of the chirping, which gives a more brief form for the local Hamiltonian. 
In the onset condition, the frequency of the wave is assumed to be time independent, and we show an integral of motion for the trapped electrons under resonant interactions.
We then show the adiabatic regime of the resonant electron phase space behavior, and verified adiabatic approximation from simulation based on our theory.


We organize our paper as follows. In section~\ref{sec:theory}, we present the Hamiltonian theory and derive the Vlasov equation for the resonant particle distribution, and the wave equation for the slowly varying wave envelope.
In section~\ref{sec:dis}, 
% and give the chirping rate at typical parameters.  
Finally, the summary is presented in section~\ref{sec:conc}.
