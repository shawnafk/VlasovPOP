\section{Conclusion}
\label{sec:conc}
In this paper, we explore the dynamics of resonant electrons and the evolution of whistler chorus wave within the Earth's inner magnetosphere, focusing on the scale separation of electron motion and fast wave interaction.
We developed a Hamiltonian theory describes the dynamics of the resonant particles on the reference frame moving with the local resonance.
The slowly varying motion along the weakly inhomogeneous magnetic field and the fast varying wave-particle interaction can be separately managed.
Our work also provides several new view angle for the chorus chirping problem.
In the onset stage of the chorus chirping, our theoretical description can be greatly reduced, which is beneficial to compose of numerical solver without losing any key physics for the generation mechanism of the chorus emission.  
The discussed adiabatic regime also shows the potential chirping behavior with respect to the inhomogeneity parameter $\alpha$.
