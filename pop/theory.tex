\section{Hamiltonian Theory in local resonance frame}
\label{sec:theory}
%Hamiltonian 
%Given the Hamiltonian and wave form (original)
%PART I: THE WAVE
Consider an electron with mass $m_e$ gyrating in a weakly inhomogeneous magnetic field.
It interacts with a  narrowband traveling wave propagating along the magnetic field line. 
%with varying phase $\phi_q$.
The wave vector potential can be represented in a Fourier integral
\begin{equation}\label{eq.def_A}
    \mathbf{A}(s,t) =  \int_{-\infty}^{\infty}\mathrm{d} q~ \mathbf{A}\left(q,t\right) \exp(\imath\phi_{q})~,
\end{equation}
where 
\begin{equation}
    \phi_q \equiv \int^t \omega(q, \tau)d\tau - q s~
\end{equation}
% $\phi_q \equiv \int^t \omega(q, \tau)d\tau - q s$
is the  wave phase,
$\omega$ is the wave frequency, $s$ is the coordinates along the field line, and $q$ is the Fourier harmonic number.
%Here, we concern with the resonant interactions along a one-dimensional magnetic field line, thus $s$ and $q$ are in scalar form, for simplicity.
%\subsection{The local Hamiltonian theory}
%PART II: THE HAMILTONIAN

Considering the perturbed wave field is much smaller than the background field, we can treat the Hamiltonian as the equilibrium part and the perturbed part.
%The former describes the particle motion in a slowly-varying magnetic field, and it is typically associated with three adiabatic invariants associated with the unperturbed equilibrium, namely the magnetic moment, the longitudinal invariant, and the flux invariants.
%They correspond to the gyro motion, the mirror bounce motion, and the drift motion perpendicular to the magnetic field.
%Since the timescale of perpendicular drift motion is very slow compare to the time period of the wave instabilities, we can neglect drift motion, and leave 
The equilibrium Hamiltonian is
\begin{equation}\label{eq.def_H0}
    H_0(\mu;s,p_\|) =  \frac{p_\|^2}{2 m_e} + \mu \omega_{ce}(s)~,
\end{equation}
where $p_\|$ is the parallel momentum, $\omega_{ce}=eB/m_ec$ is the electron gyrofrequency, and 
\begin{equation}
\mu \equiv {m_e v_\perp^2 \over 2 \omega_{ce}}.
\end{equation}


%While the latter describes the interaction with the banded wave, and 
The perturbed Hamiltonian 
due to the interaction with the wave field
can be  represented in a Fourier integral form
\begin{equation}\label{eq.def_H}
    \delta H(\varphi,\mu; s,p_\|;t) = 
    %\frac{1}{2}
     \int_{-\infty}^{\infty} \mathrm{d} q~{\delta H}\left(\mu, p_\|, q ; t\right) e^{\imath\left(\phi_{q}-\varphi\right)},
    %+ \mathrm{c.c}~,
\end{equation}
where %$Re$ denotes taking the real part 
$\varphi$ is the particle gyroangle.
The convention is that the real part of the equation is to be taken.
%$\phi_q - \varphi$ is the difference between the wave angle and the gyro angel.
%Note that the perturbed field is much less than the equilibrium field, we still regard $s$ and $\phi$ as canonical coordinates, and $p_\parallel$ and $\mu$ as canonical momentum for the total Hamiltonian.

%Why we have to study the Hamiltonian on the cell
%the canonical pertubation theory
%Conventionally, the gyrokinetics based on the non-canonical perturbation theory can be applied to eliminate the fast gyromotion scale. However, according to the above scale analysis, finding suitable coordinates for scale separation is hard.
%Instead, we propose a method using the local canonical perturbation to avoid the Jacobi calculation and separate the spatial-temporal scales based on the locality of the dynamics.
%\subsubsection{The expansion of the phase on the cell}
%The expansion of phi on the cell
The wave form and the Hamiltonian in Eqs. (\ref{eq.def_A}) and (\ref{eq.def_H}) involves wave phase $\phi_q$ and the particle gyro phase $\varphi$. Both of them change fast in the laboratory frame of reference. While the wave-particle interaction scale and the adiabatic motion scales in the equilibrium magnetic field are considerable slower. 
We here example several typical motion scales and their characteristic time and length. 
There has wave oscillation scale, as in frequency $\omega$ and wavenumber $k$;
energetic particle gyration scale, as in gyro frequency $\omega_\mathrm{ce}$ and gyro radius $r_{ce}$;
wave trapped particle bounce frequency $\omega_\mathrm{b}$; wave chirping scale, as in $\dot{\omega}/\omega$, $\nabla k/k$, and $\nabla a/a$; background spatial nonuniformity scale, as in $\nabla B/B$; particle parallel bouncing scale $\omega_B$ and $L_\mathrm{B}$; particle transverse drift scale, as in $\omega_D$ and $L_\mathrm{d}$. 
For the chorus wave in the Earth's dipole field, the relations of the characteristic scales are
\begin{equation}
    \begin{aligned}
        &\omega_\mathrm{ce} \sim \omega >  \omega_\mathrm{b} \gtrsim \dot\omega/\omega  \gg \omega_\mathrm{B} \gg \omega_\mathrm{d}~,
        \\
        &r^{-1}_{ce} \sim k \gg \nabla k/k\sim \nabla a/a \gg \nabla B/B \sim L_\mathrm{B}^{-1} \gg L_\mathrm{d}^{-1}~.
    \end{aligned}
\end{equation}
The last scale corresponds to the third adiabatic invariant, which is well-conserved in our studies. 
In contrast, the second to last one corresponds to the longitudinal invariant and is not constant of motion due to the perturbation in our study.  

Our Hamiltonian theory aim to separate those motion scales, and focus on the resonant wave-particle interaction in such circumstance.  
Note that the resonant interaction indicates the particle and the wave stay enduringly, and the motion scale seen by the particle is considerably slow. 
Therefore, if we choose a reference frame moving in the local resonant velocity, the resonance particles are at rest and see a nearly constant wave phase.
For the inhomogeneous magnetic field, we can apply the local reference frames for a discretized finite length subdomains, referred as the ``cell''.
We divide the continuous $s$ into numerous cells, centering at $s_i$ where $i$ is the index of the cell. The cell moves in the resonance velocity, i.e.,
\begin{equation}\label{eq.resonance}
    %for 1st cyclone resonance
    \frac{\mathrm{d} s_i}{\mathrm{d}t} = v_r(s_i) \equiv \frac{\omega_{i}\left(k_{i}(t), t\right)-\omega_{ce}\left(s_{i}(t)\right)}{k_{i}(t)}~,
\end{equation}
Each cell represents a local resonance reference frame, and within a cell, we can locally expand the related quantities with respect to the cell center.
%The phase $\phi_q$ is then splits into the fast and slowly varying terms, i.e., $\phi_q \simeq \phi_{f} + \phi_{s}$.
%del . The spectrum broadening due to frequency chirping is small within the resonance cell. Thus,We recall the fast varying phase

We assume that the wave spectrum is locally peaked at $k_i(s_i,t)$,
%If the finite wave train is not too broad in its wave-number spectrum, 
then the frequency can be expanded around  $k_i(s_i,t)$ as 
%For the triggering frequency chirping emission, although the background wave has a broad spectrum, the unstable mode is narrow in spectrum space.
%Thus, We treat the wave spectrum as locally centered at $k_i(s_i,t)$.along the magnetic field line, and we can expand the phase with respect to the cell center. The Taylor expansion of the frequency in the first term up to the first-order is
\begin{equation}
    \omega(q,\tau) \simeq \omega_{i}\left(k_{i}(\tau), \tau\right) + \frac{\partial \omega}{\partial k_{i}}\left(q-k_{i}(\tau)\right)~.
\end{equation}
We further assume that the width of cell $\Delta s$ is enough small, so that 
%the phase difference due to finite spreading of spectrum satisfies 
$\left(q-k_i(t)\right) \Delta s \ll 1$. Then 
we 
%further expand
the term $q s$ as 
%and keep the terms up to the first-order,
\begin{equation}
    \begin{aligned}
        q s & = (k_i + (q-k_i)) (s_i+(s-s_i))           \\
            & \simeq  q s_i + k_i (s-s_i).                
            % \\
            %& = q \int^t v_r(\tau) d \tau  + k_i (s-s_i)~.
    \end{aligned}
\end{equation}
%The derivation is under the condition
Then the expansion of wave phase $\phi_q$ about $s_i$ and $k_i$ yields
\begin{equation}
\begin{aligned}
    \phi_{q} 
    & \simeq  \int^{t}  d \tau \left( \omega_{c e}\left(s_{i}(\tau)\right)+ k_i(\tau) v_r(\tau)  + \left(q-k_{i}(\tau)\right)\frac{\partial \omega}{\partial k_{i}} \right) 
        \\ 
        &- q \int^t v_r(\tau) d \tau - k_{i}(t)\left(s-s_{i}(t)\right)~,
\end{aligned}
\end{equation}
%Applying 
where 
the cell coordinate has been written as $s_i= q \int^t v_r(\tau) d \tau$
and
the  cyclotron resonance condition has been applied at the cell center
\begin{equation}
    \omega_i - k_i v_r - \omega_{ce} = 0~.
\end{equation}
%and replacing $\omega_i$, we have
%\lipsum[1]
%\begin{widetext}
%\[
%    \phi_q \simeq \int^{t}  d \tau \left( \omega_{c e}\left(s_{i}(\tau)\right)+ k_i(\tau) v_r(\tau)  + \left(q-k_{i}(\tau)\right)\frac{\partial \omega}{\partial k_{i}} \right) - q \int^t v_r(\tau) d \tau - k_{i}(t)\left(s-s_{i}(t)\right)~.
%\]
%\end{widetext}
%\lipsum[1]
% \begin{equation}
% \begin{aligned}
%         \phi_q &\simeq \int^{t}  d \tau \left( \omega_{c e}\left(s_{i}(\tau)\right)+ k_i(\tau) v_r(\tau)  + \left(q-k_{i}(\tau)\right)\frac{\partial \omega}{\partial k_{i}} \right) 
%         \\ 
%         &- q \int^t v_r(\tau) d \tau - k_{i}(t)\left(s-s_{i}(t)\right)~.
% \end{aligned}
%\end{equation}
%It is clear that 
Then the wave phase can be split into the fast and slowly varying parts,
\begin{equation}\label{eq.phi_fs}
    \begin{aligned}
        \phi_q & \simeq \phi_{f} + \phi_{s},
        \\
        \phi_f & = \int^{t} \omega_{ce} \left(s_{i}(\tau)\right) d \tau -k_{i}(t)\left(s-s_{i}(t)\right),
        \\
        \phi_s & = \int^{t}\left(q-k_{i}(\tau)\right)\left(\frac{\partial \omega}{\partial k_{i}}-v_{i}(\tau)\right) d \tau~,
    \end{aligned}
\end{equation}
%For the slowly varying chirping wave,
where the term $\phi_f$ represents the fast varying wave phase inside the envelope
and the term 
$\phi_s$ describes the slow variation of the wave kernel  observed in the cell reference frame.
% which is related to the changing wave number and the velocity difference, i.e., $\int^t\Delta v \Delta k \mathrm{d} \tau$.On the other hand, the term $\phi_f$ is qualitatively equal to $\omega_i t - k_i s_i$ is the fast varying wave phase inside the envelope.

%The Hamiltonian in Eq.~(\ref{eq.kernel_H}) and the wave field in Eq.~(\ref{eq.kernel_A}) have been transformed into the product of the local slowly-varying kernel on $s_i$ and the fast-varying phase.
To obtain the Hamiltonian in the local resonance frame, 
%we have to change the Hamiltonian on the global domain to that represented on each individual local reference frames.
%The form of the separated phase hints us to 
we construct a  canonical transformation given by a generating function of the second kind,
%F(q,P)
%k_{i}(t)\left(s-s_{i}(t)\right)+ \varphi-\int^{t} \omega_{ce}\left(s_{i}(\tau)\right)\mathrm{d}\tau
\begin{equation}\label{eq.gf2}
\begin{aligned}
        F_{i}(s, \varphi ; \Omega, \mathcal{J};t) & = \left(\varphi - \phi_f \right) \left(\Omega+\Pi_{i}(t)\right)  +
        \\
        &\left(\varphi-\int^{t} \omega_{ce}\left(s_{i}(\tau)\right)\mathrm{d}\tau\right)\mathcal{J}~,
\end{aligned}
\end{equation}
where 
%$s$ and $\varphi$ are the old canonical coordinates, while 
$\Omega$ and $\mathcal{J}$ are the new canonical variables
%momentum of the new Hamiltonian. 
and $\Pi_i\equiv\Pi_i(t)$
is a  function
% only depends only on
of time to be determined.
% We will show that the choice of $\Pi_i(t)$ depends on the resonance velocity.
%, which converts the dynamics to each local resonance reference frames.
%It is set according to the requirement of the new Hamiltonian. 
According to the definition of the generating function \cite{goldstein2001}, 
%we can obtain 
the new canonical coordinates are 
%k_{i}(t)\left(s-s_{i}(t)\right)+\varphi-\int^{t} \omega_{c e}\left(s_{i}(\tau)\right) d \tau =
\begin{equation}\label{eq.newQ}
    \begin{aligned}
         \xi & \equiv \frac{\partial F_i}{\partial \Omega} =  \varphi - \phi_f~,
         \\ 
        \vartheta & \equiv   \frac{\partial F_i}{\partial \mathcal{J}} = \varphi -\int^{t} \omega_{c e}\left(s_{i}(\tau)\right) d \tau ~,
    \end{aligned}
\end{equation}
The generating function also gives the original canonical momentum
\begin{equation}\label{eq.oldP}
    \begin{aligned}
        p_\| & \equiv \frac{\partial F_i}{\partial s} = k_{i}(t)( \Omega+\Pi_{i})~,
        \\
        \mu & \equiv \frac{\partial F_i}{\partial \varphi} = \mathcal{J} +\Omega+\Pi_{i}~,
    \end{aligned}
\end{equation}
Thus the new canonical variables are
\begin{equation}\label{eq.newP}
    \begin{aligned}
        \Omega & = \frac{p_{\|}}{k_{i}(t)}-\Pi_{i}(t)~,
        \\
        \mathcal{J}   & = \mu-\Omega-\Pi_{i}(t)=\mu-\frac{p_{\|}}{k_{i}(t)}~.
    \end{aligned}
\end{equation}
%The new variable $\xi,~\Omega$ denoting the particle gyro phase and parallel momentum at the cell frame of reference.
%For the homogeneous magnetic field, $\vartheta$ is called the helical angle, and $\mathcal{J}$ is the conjugated helical invariant.
%where the canonical coordinate $\xi$ is the fast varying variable carrying with the wave fluctuations and $s_i(t)$ is the slowly varying variable carrying with the wave envelope during the wave-particle interaction.

The generating function in Eq.~(\ref{eq.gf2}) is unique for each local cell, it readily changes the Hamiltonian in the laboratory frame to the local resonance frame.
The Hamiltonian with the new canonical variables $(\vartheta,\mathcal{J},\xi,\Omega)$ on the cell can be obtained from 
\begin{equation}\label{eq.HamiltonianK}
    H_{i} (\xi, \Omega, \vartheta, \mathcal{J} ; t)=H_{0}+ \delta H+\frac{\partial F_{i}}{\partial t}~.
\end{equation}


With Eqs. (\ref{eq.phi_fs}) 
and (\ref{eq.newQ})
the phase of the perturbed Hamiltonian
%, according to the definition in 
Eq. (\ref{eq.def_H}) is written  in terms of the new canonical angle
%, we first separate the phase $\phi_q$ according to equation (\ref{eq.phi_fs}), and obtain
\begin{equation}\label{eq.kernel_H}
    \begin{aligned}
        \delta H(\varphi,\mu;s,p_\|;t)  \simeq \delta H_i e^{\imath\left(\phi_{f}-\varphi\right)}=\delta H_i e^{-\imath \xi}~
    \end{aligned}
\end{equation}
where  
\begin{equation}
        \delta H_i\left(\mu, p_\|, s_i; t\right)=\int_{-\infty}^{\infty} \mathrm{d} q~ \delta H\left(\mu, p_\|, q ; t\right)  e^{\imath \phi_{s}}. 
\end{equation}
%The real part of the perturbed Hamiltonian $\delta H\left(\mu, p_\|, s_i; t\right)$ 
The magnitude of the perturbed Hamiltonian due to the wave-particle interaction  
is given in terms of the  wave envelope as 
%In the new canonical coordinates it becomes
\begin{equation}
\delta H_i={e \over c} v_{\perp}|a\left(s_{i}, t\right)| = m_e \frac{\omega_b^2}{k_i^2}~,
\end{equation}
where the trapped particle bounce frequency 
$\omega_b^2 \equiv e \delta B  k_i v_\perp / m_e c$ in previous studies 
\cite{omura_theory_2008,sudan_theory_1971,tao_theoretical_2020} 
%now satisfies
is defined as
\begin{equation}
    \omega_b^2 %= \frac{e}{m_e c}\delta B  k_i v_\perp 
    = \frac{k_{i}^2 }{m_e } {e\over c} |a(s_{i}, t)| v_\perp  
\end{equation}
with  
the perpendicular velocity written in terms of the new canonical variables Eq. (\ref{eq.oldP}),
\begin{equation}
 v_\perp =\sqrt{\frac{2\omega_{ce}\left(s_{i}\right)\left(\mathcal{J}+\Omega+\Pi_{i}\right)}{m_e}} ~.
\end{equation}

% The phase term of the perturbed Hamiltonian, simply written in new canonical angle as $\exp({\imath\left(\phi_{f}-\varphi\right)}) = \exp{(- \imath \xi)}$. 
% The final form of the perturbed Hamiltonian is 
% \begin{equation}
%         \delta H_i\left(\Omega, J; s_{i}, t\right) =
%         % \frac{1}{2}+ \mathrm{c.c}
%         Re \left(\delta H_i e^{-\imath \xi}\right) = m_e \frac{\omega_b^2}{k_i^2} \cos\xi~.
% \end{equation}



The equilibrium Hamiltonian Eq.~(\ref{eq.def_H0})
%in the first term 
in the new canonical variables Eq. (\ref{eq.oldP}) 
becomes
\begin{equation}\label{eq.ctH0}
     H_0 = \frac{k_{i}^{2}(t)\left(\Omega+\Pi_{i}\right)^{2}}{2 m_{e}}+\left(\mathcal{J}+\Omega+\Pi_{i}\right) \cdot \omega_{c e}(s). 
\end{equation}
The expansion of gyrofrequency about the cell center $s_i$ yields
\begin{equation}
\omega_{c e}(s) \simeq \omega_{c e}(s_i) +  \frac{\partial \omega_{ce}}{\partial s} (s-s_i).
\end{equation}
The displacement from the cell center can be written as 
$s - s_{i}(t) = (\xi-\vartheta)/k_{i}(t)$ according to the definition of the canonical variables in Eq.~(\ref{eq.newQ}). 
Then we have 
\begin{equation}\label{eq.H0}
\begin{aligned}
  H_0 &=\frac{k_{i}^{2}(t) \Omega^{2}}{2 m_{e}}+\frac{k_{i}^{2}(t)}{m_{e}} \Pi_{i} \Omega+(\mathcal{J} + \Omega + \Pi_i)\omega_{ce}(s_i)
  \\
  &+\left[\left(\mathcal{J} +\Omega+\Pi_{i}\right)\frac{\mathrm{d}\omega_{c e}}{\mathrm{d}s_{i}} \right]\left(\frac{\xi-\vartheta}{k_{i}(t)}\right) +\frac{k_i^2}{2m_e} \Pi_i^2.
    \end{aligned}
\end{equation}
%where the term of $ \Pi_i^2$ has been neglected. 
%The time dependent function $\Pi_i(t) \equiv (m_e / k_i) \cdot \mathrm{d}{s}_i/\mathrm{d}t$
The time derivative of the generating function yields
\begin{equation}\label{eq.pFpt}
    \begin{aligned}
    \frac{\partial F_i}{\partial t} & =- k_i \frac{\mathrm{d} s_i}{\mathrm{d} t}\Omega +\frac{\mathrm{d} k_i}{\mathrm{d} t}(\Omega + \Pi) \frac{\xi-\vartheta}{k_i(t)} 
    \\
    & + \frac{\mathrm{d} \Pi_i}{\mathrm{d} t} \xi - (\mathcal{J} + \Omega + \Pi_i)\omega_{ce}(s_i)- k_i \frac{\mathrm{d} s_i}{\mathrm{d}t}\Pi_i. 
    \end{aligned}
\end{equation}
In view of Eqs.~(\ref{eq.H0}) and (\ref{eq.pFpt}) there are two linear terms of the canonical momentum $\Omega$ 
in the new Hamiltonian: $$\left(\frac{k_{i}^{2}(t)}{m_{e}} \Pi_{i}-k_i(t) \frac{\mathrm{d} s_i}{\mathrm{d} t}\right) \Omega $$
and 
$$\left(\frac{\mathrm{d}k_{i}}{\mathrm{d}t}+\frac{\mathrm{d}\omega_{c e}}{\mathrm{d}s_{i}}\right)\Omega.$$
We then eliminate the  linear terms of the canonical momentum $\Omega$ by setting 
\begin{equation}\label{eq.PI}
    \Pi_i(t) = \frac{m_e}{k_i(t)}\frac{\mathrm{d} s_i}{\mathrm{d} t} = \frac{m_e}{k_i^2(t)}(\omega_i - \omega_{ce}(s_i))~.
\end{equation}
Consequently, we have $\Omega \approx 0$ for particles near the resonance with $p_{||}\simeq m_e v_r$.
Thus the second linear term of $\Omega$ can be neglected whenever the background parameters change slowly 
during a bounce period $\omega_b^{-1}$ of the particle trapped in the  wave field, 
\begin{equation}
    \frac{\mathrm{d} k_i}{\mathrm{d} t} + \frac{\partial \omega_{ce}}{\partial s_i} \ll k_i(t) \omega_b~.
\end{equation}
With the $\Pi_i$ defined in Eq.~(\ref{eq.PI}), we have
\begin{equation}
    \frac{\mathrm{d}\Pi_i}{\mathrm{d}t} = \frac{m_e}{k_i(t)}\frac{\mathrm{d}s_i^2}{\mathrm{d t^2}} - \frac{\mathrm{d}k_i}{\mathrm{d t}}\frac{\Pi_i}{k_i}~.
\end{equation}
Then the sum of Eqs.~(\ref{eq.H0}) and (\ref{eq.pFpt}) becomes
\begin{equation}\label{eq.merge}
    \begin{aligned}
        H_0 + \frac{\partial F_i}{\partial t}& \simeq \frac{k_{i}^{2}(t) \Omega^{2}}{2 m_{e}} 
        + \left[m_{e} \frac{d^{2} s_{i}}{\mathrm{d}t^{2}}+\frac{\mathrm{d}\omega_{c e}}{\mathrm{d}s_{i}}\left(\mathcal{J}+\Pi_{i}\right)\right] \frac{\xi}{k_{i}(t)} 
        \\
        &-\left[\left(\frac{\mathrm{d}k_{i}}{\mathrm{d}t}+\frac{\mathrm{d}\omega_{c e}}{\mathrm{d}s_{i}}\right) \Pi_{i}+\frac{\mathrm{d}\omega_{c e}}{\mathrm{d}s_{i}} \mathcal{J}\right] \frac{\vartheta}{k_{i}(t)} ~.
    \end{aligned}
\end{equation}
where the  $ \Pi_i$ terms without the canonical variables 
have been neglected.



% Collecting the terms in Eq.~(\ref{eq.ctH0}) and Eq.~(\ref{eq.pFpt}), we have
% \begin{equation}\label{eq.merge}
%     \begin{aligned}
%     H_0 + \frac{\partial F_i}{\partial t} & = \frac{k_{i}^{2}(t) \Omega^{2}}{2 m_{e}}+\left(\frac{k_{i}^{2}(t)}{m_{e}} \Pi_{i}-k_i(t) \frac{\mathrm{d} s_i}{\mathrm{d} t}\right) \Omega +\frac{\mathrm{d}\Pi_{i}}{\mathrm{d}t} \xi 
%     \\
%     & +\left[\left(\frac{\mathrm{d}k_{i}}{\mathrm{d}t}+\frac{\mathrm{d}\omega_{c e}}{\mathrm{d}s_{i}}\right)\left(\Omega+\Pi_{i}\right)+\frac{\mathrm{d}\omega_{c e}}{\mathrm{d}s_{i}} J\right]\left(\frac{\xi-\vartheta}{k_{i}(t)}\right)
%     \\
%     &+ \frac{k_i^2}{2m_e} \Pi_i^2 - k_i \frac{\mathrm{d} s_i}{\mathrm{d}t}\Pi_i ~.
%     \end{aligned}
% \end{equation}
% Note that the last two terms in the above equation can be neglected.
% Those terms are canonical variables free, i.e., they do not contribute to the dynamics and can be eliminated readily by introducing an arbitrary time dependent function in the $F_i$ in the first place. 



% Substituting it into Eq.~(\ref{eq.merge}), we have 
% \begin{equation}
%     \begin{aligned}
%         H_0 + \frac{\partial F_i}{\partial t} & \approx-\left[\left(\frac{\mathrm{d}k_{i}}{\mathrm{d}t}+\frac{\mathrm{d}\omega_{c e}}{\mathrm{d}s_{i}}\right) \Pi_{i}+\frac{\mathrm{d}\omega_{c e}}{\mathrm{d}s_{i}} J\right] \frac{\vartheta}{k_{i}(t)} 
%         \\
%          & +\frac{k_{i}^{2}(t) \Omega^{2}}{2 m_{e}}+\left[m_{e} \frac{d^{2} s_{i}}{\mathrm{d}t^{2}}+\frac{\mathrm{d}\omega_{c e}}{\mathrm{d}s_{i}}\left(J+\Pi_{i}\right)\right] \frac{\xi}{k_{i}(t)} 
%     \end{aligned}
% \end{equation}

%Combining the equilibrium and   we have
% \begin{equation}\label{eq.H_all_0}
%     \begin{aligned}
%         H_i &\left(\xi, \Omega;\vartheta, J; s_{i}, t\right) = \frac{k_{i}^{2}(t) \Omega^{2}}{2 m_{e}} + m_e \frac{\omega_b^2}{k_i^2} \cos\xi
%         \\
%         &+ \left[m_{e} \frac{d^{2} s_{i}}{\mathrm{d}t^{2}}+\frac{\mathrm{d}\omega_{c e}}{\mathrm{d}s_{i}}\left(J+\Pi_{i}\right)\right] \frac{\xi}{k_{i}(t)} 
%         \\
%         &-\left[\left(\frac{\mathrm{d}k_{i}}{\mathrm{d}t}+\frac{\mathrm{d}\omega_{c e}}{\mathrm{d}s_{i}}\right) \Pi_{i}+\frac{\mathrm{d}\omega_{c e}}{\mathrm{d}s_{i}} J\right] \frac{\vartheta}{k_{i}(t)} ~.
%     \end{aligned}
% \end{equation}
%def alpha first
% We can further simplify $\alpha$ by expanding the second order derivative of $s_i$ to show the physics meaning of $\alpha$, 
% \begin{equation}
%     \begin{aligned}
%     \frac{\mathrm{d^2}s_i}{\mathrm{d}t^2} &= \frac{\mathrm{d}v_r}{\mathrm{d}t} = \frac{\mathrm{d}}{\mathrm{d}t} \frac{\omega_i - \omega_{ce}}{k_i}
%     \\
%     & = \frac{1}{k_i}\frac{\mathrm{d}(\omega_i - \omega_{ce})}{\mathrm{d}t} - \frac{\omega_i - \omega_{ce}}{k_i^2}\frac{\mathrm{d}k_i}{\mathrm{d}t}~.
%     \end{aligned}
% \end{equation}
% The exact derivatives are given along the resonant trajectory, i.e.,
% \begin{equation}
%     \frac{\mathrm{d}}{\mathrm{d}t} \equiv \frac{\partial }{\partial t} + v_r \frac{\partial }{\partial s}~.
% \end{equation}
% For the frequency $\omega_i$, applying the identity
% \begin{equation}
%     \frac{\partial \omega_i}{\partial t} + v_g \frac{\partial \omega_i}{\partial s} = 0
% \end{equation}
% we have 
% \begin{equation}
%     \frac{\mathrm{d}\omega_i}{\mathrm{d}t} = \left(1-\frac{v_r}{v_g}\right) \frac{\partial \omega_i}{\partial t}~.
% \end{equation}
% For the derivative of wave number $k_i$, we additionally use
% \begin{equation}
%     \frac{\partial k_i}{\partial t} \equiv - \frac{\partial \omega_i}{\partial s},
% \end{equation}
% and have
% \begin{equation}
%     \frac{\mathrm{d}k_i}{\mathrm{d}t} = \frac{1}{v_g}\frac{\partial \omega_i}{\partial t} + v_r \frac{\partial k_i}{\partial s}~.
% \end{equation}
% The time derivative of gyrofrequency only depends on $s$, and
% \begin{equation}
%     \frac{\mathrm{d}\omega_{ce}}{\mathrm{d}t} = v_r \frac{\partial \omega_{ce}}{\partial s}~.
% \end{equation}
% Substituting back to equation (\ref{eq.alp0}), we have the final expression of $\alpha$
% \begin{equation}\label{eq.alp0.5}
%     \alpha \equiv \frac{1}{\omega_{b}^2}\left[\left(1 - 2\frac{v_r}{v_g}\right)\frac{\partial \omega_i}{\partial t}  -v_r^2 \frac{\partial k_i}{\partial s_i}+ \frac{\mathrm{\partial} \omega_{ce}}{\mathrm{\partial} s_{i}}\frac{k_i}{m_e}\mathcal{J}\right].
% \end{equation}
%With the introduced parameter, we write 
%can  organize the as 
We further  introduce a dimensionless parameter that describes the particle trapping,
%$\alpha$ from shown in equation 
\begin{equation}\label{eq.alp0}
    \alpha = \frac{1}{\omega_b^2}\left[k_i \frac{\mathrm{d}^{2} s_{i}}{\mathrm{d} t^{2}}+ \frac{k_i}{m_{e}} \frac{d \omega_{ce}}{d s_{i}}\left(\mathcal{J}+\Pi_{i}\right)\right]~.
\end{equation}
Combining Eqs. (\ref{eq.kernel_H}) and (\ref{eq.merge}) we obtain the total Hamiltonian  
Eq. (\ref{eq.HamiltonianK})
in the resonance frame
\begin{equation}\label{eq.HamiltonianK2}
    \begin{aligned}
    H_i(s_i,\vartheta,\mathcal{J},\xi,\Omega) &= \frac{k_{i}^{2}\Omega^{2}}{2 m_{e}} + m_e \frac{\omega_b^2}{k_i^2} ( \cos \xi + \alpha \xi) 
    \\
    &- \left[\left(\frac{\mathrm{d} k_{i}}{\mathrm{d} t}+\frac{d \omega_{ce}}{d s_{i}}\right) \Pi_{i}+\frac{d \omega_{ce}}{d s_{i}} \mathcal{J}\right] \frac{\vartheta}{k_{i}}~,
    \end{aligned}
\end{equation}
where the real part of perturbed Hamiltonian Eq. (\ref{eq.kernel_H}) has been taken.
Note that the Hamiltonian Eq.~(\ref{eq.HamiltonianK2}) is 
%The Hamiltonian composed 
a modified nonlinear pendulum  system
and
the motion scales have been separated in the Hamiltonian.
The first two terms in Eq.~(\ref{eq.HamiltonianK2}) describe the fast varying scale in the cell
and 
the $\alpha$ term acts as a noninertial force arising from background magnetic field inhomogeneity.
%It includes the mirror force from inhomogeneous magnetic field and force from wave chirping.
%The parameter itself, is a dimensionless parameter representing the ratio of the inertial force and wave restoring force, i.e., the $\omega_b$.
%For the trapped particles, the ratio is from $-1$ to $1$.
%The sign of $\alpha$ indicates the orientation of the  hole.
When $\alpha = 0$, Eq.~(\ref{eq.HamiltonianK2}) is reduced to a simple pendulum Hamiltonian, which describes the particle oscillation in the wave potential well.
When $|\alpha|$ approaches to 1, it indicates that the inertial force destroys the entire wave trapping. 
%invalidate the adiabatic theory.
When $|\alpha|>1$, trapped particles would not exist.

%--------- statement is not true for our work
%The perturbation theory let us apply $J$ and $\vartheta$ obtained from the equilibrium Hamiltonian equation, i.e., the zeroth order solution, in the canonical equation of the perturbed Hamiltonian to calculate the variation of $\Omega$ and $\xi$.
%On the contrary, the last term of Eq.~(\ref{eq.HamiltonianK2}) containing $\vartheta$ determines the evolution of momentum $\mathcal{J}$ along unperturbed particle orbit. The term mainly depends on the equilibrium parameters, and yields
%The last term of Eq.~(\ref{eq.HamiltonianK2}) containing $\vartheta$ which gives the evolution of momentum $\mathcal{J}$ from the canonical 
With Eq.~(\ref{eq.HamiltonianK2}) Hamilton's equations  give 
\begin{equation}\label{eq.m3}
        \frac{d \mathcal{J}}{d t} =-\frac{\partial H_{i}}{\partial \vartheta}=\frac{1}{k_{i}(t)}\left[\left(\frac{d k_{i}}{d t}+\frac{d \omega_{c e}}{d s_{i}}\right) \Pi_{i}+\frac{d \omega_{c e}}{d s_{i}} \mathcal{J}\right]~.
\end{equation}
%------------our work is different from the pertubation theory
%and according to the perturbation theory, we can neglect the contribution from the perturbed Hamiltonian and have
%In fact, we can neglect the contribution from the perturbed Hamiltonian and have
and
\begin{equation}\label{eq.m4}
    \frac{d \vartheta}{d t} =\frac{\partial H_{i}}{\partial \mathcal{J}} \simeq \frac{d \omega_{c e}}{d s_{i}} \frac{\xi-\vartheta}{k_{i}(t)}=\frac{d \omega_{c e}}{d s_{i}}\left(s-s_{i}(t)\right)~.
\end{equation}
%the onset approximation, copy from cpc paper
%The canonical equation for the perturbed wave-particle interaction is then given as 
Similarly, Hamilton's equations  give
\begin{equation}\label{eq.m1}
    \frac{\mathrm{d}\Omega}{\mathrm{d}t} = - \frac{\partial H_i}{\partial \xi} = m_e \frac{\omega_b^2}{k_i^2}\left(\sin \xi - \alpha \right)
\end{equation}
and 
%the evolution of $\xi$ is 
\begin{equation}\label{eq.m2}
    \frac{\mathrm{d}\xi}{\mathrm{d}t} = \frac{\partial H_i}{\partial \Omega} =\frac{k_{i}^{2}\Omega}{ m_{e}}+ \frac{\sqrt{2\omega_{ce}} }{2\sqrt{m_e (\mathcal{J}+\Omega+\Pi_i)}}\frac{e |a|}{c}(\cos \xi + \alpha \xi)~.
\end{equation}
Equations~(\ref{eq.m1})-(\ref{eq.m4}) gives the evolution of the dynamics system inside the cell and with the cell. 

%\subsection{The Vlasov equation and wave envelope on the cell}
%The full form of the Vlasov equation
According to the Hamiltonian theory for dynamics of the resonance particle on the cell reference frame, we are now able to construct the Vlasov and the wave envelope equations which self-consistently describes the interaction of the resonant particles with the slowly varying wave envelope and the evolution of the wave.

The distribution function of resonant electrons is described separately on difference cells.
Thus, the distribution function depends on the cell coordinate $s_i$.
Besides, the dynamics concerning the angle variable $\vartheta$ in Eq.~(\ref{eq.m4}) can be neglected, since
\begin{equation}
   \frac{\mathrm{d} \omega_{ce}}{\mathrm{d} s_{i}}\left(s-s_{i}(t)\right) \ll \omega_b
\end{equation}
during the onset stage of the chorus emission.
Thus, the distribution function depends on canonical variables $\mathcal{J}$,$\xi$, and $\Omega$ only, i.e., $f(s_i,\mathcal{J},\xi,\Omega)$.
Combining the canonical Hamiltonian equations, we can directly write the Vlasov equation as 
\begin{equation}\label{eq.vlasov}
    %\frac{\partial f}{\partial t}+ \frac{d s_{i}}{d t} \frac{\partial f}{\partial s_{i}} - \frac{\partial H}{\partial \vartheta} \frac{\partial f}{\partial \mathcal{J}} + \frac{\partial H}{\partial \Omega} \frac{\partial f}{\partial \xi} - \frac{\partial H}{\partial \xi} \frac{\partial f}{\partial \Omega}=0~.
    \frac{\partial f}{\partial t}+ \frac{d s_{i}}{d t} \frac{\partial f}{\partial s_{i}}
+ \left[f, H\right]_{\vartheta,\mathcal{J}} +  \left[ f, H\right]_{\xi,\Omega}=0~.
    % - \frac{\mathrm{d} \mathcal{J}}{\mathrm{d}t} \frac{\partial f}{\partial \mathcal{J}} + \frac{\mathrm{d}\xi}{\mathrm{d} t} \frac{\partial f}{\partial \xi} - \frac{\mathrm{d}\Omega}{\mathrm{d} t} \frac{\partial f}{\partial \Omega}
\end{equation}
where the Poisson brackets are defined as
\begin{equation}
    [f,~g]_{x,y} = \frac{\partial f}{\partial x}\frac{\partial g}{\partial y}-\frac{\partial f}{\partial y}\frac{\partial g}{\partial x}~.
\end{equation}



The time derivatives of $s_i$ is given according to the local resonant velocity,
\begin{equation}
    \frac{\mathrm{d}s_i}{\mathrm{d}t} = \frac{\omega_i- \omega_{ce}}{k_l}.
\end{equation}
The variation of $\mathcal{J}$ is obtained from equation (\ref{eq.Jcons}).
The fast varying scale motions are given by equation (\ref{eq.m1}) and (\ref{eq.m2}) with simplified $\alpha$ in equation (\ref{eq.alp1}).

%The separation of the temporal and spatial scales spontaneously appears in the distribution function $f_i(\xi, \Omega, J; s_i, t)$ due to the local canonical transformation we choose, where the canonical coordinate $\xi$ is the fast varying variable carrying with the wave fluctuations and $s_i(t)$ is the slowly varying variable carrying with the wave envelope during the wave-particle interaction. The equilibrium distribution function is obtained from the adiabatic invariants $\mu$ and $J_{B}$.

%Wave
\section{wave envelope equation}
%The dispersion relation of the whistler waves is 
 %excited from the background cold plasma, where the plasma density and the magnetic field determines the cold whistler-mode wave numbers and frequencies. 
The whistler waves 
%then being 
are driven unstably by energetic electrons
 and 
the most unstable wave 
 arises from the noise level
 where the dispersion relation of the whistler wave
%with the wave number  and frequency 
is determined by 
the  cold plasma density and the background magnetic field $\mathbf{B}_0$.
%$k_l$$\omega_l$
Transverse waves propagating parallel to the direction of 
$\mathbf{B}_0$
%is electromagnetic and 
has a circular polarization. 
%considered to be fully electromagnetic, i.e., $E \perp k$ and $B \perp k$, and has a pure circular polarization, thus all field components are presented in complex form,  In the following discussion, 
Thus the wave vector potential and plasma current are represented in a complex form,
% to indicate the circular vectors, i.e., 
$A = A_x + \imath A_y$ and $J = J_x + \imath J_y$,
where the real and imaginary parts represent the orthogonal components
%x-axis direction and y-axis direction of the plane 
perpendicular to $\mathbf{B}_0$.
The evolution of the transverse wave is governed by the Ampere's law, 
\begin{equation}
    \begin{aligned}
        \label{eq.wavemid_1}
        \frac{1}{c^{2}} \frac{\partial^{2} A}{\partial t^{2}}-\frac{\partial^{2} A}{\partial s^{2}} & =\frac{4 \pi}{c}\left(J_w+J_p\right),
    \end{aligned}
\end{equation}
where the current includes the linear current $J_w$ from the bulk cold plasma and the current $J_p$ from energetic electrons.
%the cell variabel
%Similar to the Hamiltonian on the cell in Eq. (\ref{eq.kernel_H}), 
%Applying the phase expansion, 
Using Eq. (\ref{eq.phi_fs}) we  write the wave vector potential  Eq. (\ref{eq.def_A}) as 
%envelope form,
\begin{equation}\label{eq.kernel_A}
    \begin{aligned}
    A(s,t)% &\equiv \int_{-\infty}^{\infty}\mathrm{d} q~ A\left(q,t\right) \exp(\imath\phi_{q})
    %\\
    %& \simeq e^{\imath \phi_f} \int_{-\infty}^{\infty}\mathrm{d} q~ A\left(q,t\right) \exp(\imath\phi_{s})
    %\\
    %& \simeq e^{-\imath (\xi-\varphi)}  a(s_i, t)~,
    \simeq e^{\imath \phi_f(s,t)}  a(s_i, t)~,
    \end{aligned}
\end{equation}
where 
\begin{equation}
 a(s_i, t) \equiv \int_{-\infty}^{\infty} \mathrm{d} q~A(q, t)
\exp(\imath\phi_{s})
\end{equation}
 is the slowly varying wave envelope.
Similarly,
% to the slowly varying kernel in Eq. (\ref{eq.kernel_A})
the  current takes the form,
\begin{equation}\label{eq.kernel_j}
J(s,t)\simeq e^{\imath \phi_f(s,t)}  j(s_i, t)~,
%j_{p}(s_i,t) = \exp(\imath(\xi-\varphi))j_{p}(s,t)~.
\end{equation}
where  $j(s_i,t)$ is the current envelope on the cell.  
%Substituting 

Using Eq. (\ref{eq.kernel_A}) 
%into the wave equation (\ref{eq.wavemid_1}), 
we have 
\begin{equation}
    \begin{aligned}
    \frac{\partial^2  A}{\partial t^2} &\simeq %\frac{\partial }{\partial t}\left(\frac{\partial }{\partial t}(a e^{\imath \phi_f})\right) = \frac{\partial }{\partial t}\left(\frac{\partial a}{\partial t} e^{\imath \phi_f} + \imath a \frac{\partial \phi_f}{\partial t}e^{\imath \phi_f} \right)
    %\\
    %& = \frac{\partial^2 a}{\partial t^2} e^{\imath \phi_f} + \frac{\partial a}{\partial t} (\imath \frac{\partial \phi_f}{\partial t} e^{\imath \phi_f}) 
    %\\
    %& +  \imath \left(\frac{\partial a}{\partial t} \frac{\partial \phi_f}{\partial t}e^{\imath \phi_f} + a \frac{\partial ^2 \phi_f}{\partial t^2}e^{\imath \phi_f} + \imath a \left(\frac{\partial \phi_f}{\partial t}\right)^2e^{\imath \phi_f}  \right)
     %\\
     e^{\imath \phi_f} \left(\frac{\partial^2 a}{\partial t^2} + 2 \imath \frac{\partial a}{\partial t}\frac{\partial \phi_f}{\partial t} %+ \imath a \frac{\partial^2 \phi_f}{\partial t^2} - a \left(\frac{\partial \phi_f}{\partial t}\right)^2
     \right),
    \end{aligned}
\end{equation}
where the terms with 
$\partial^2 \phi_f/\partial t^2$ and $(\partial \phi_f/\partial t)^2$
%the second-order derivatives of $\phi_f$ 
have been neglected. %assumed to vanish.
%According to the definition of fast varying phase in 
Similarly, we have
%the second order derivative of $A$ with respect to $s$ is 
\begin{equation}
    \frac{\partial^2  A}{\partial s^2}  \simeq e^{\imath \phi_f} \left(\frac{\partial^2 a}{\partial s^2} + 2 \imath \frac{\partial a}{\partial s}\frac{\partial \phi_f}{\partial s} %+ \imath a \frac{\partial^2 \phi_f}{\partial s^2} - a \left(\frac{\partial \phi_f}{\partial s}\right)^2
    \right).
\end{equation}
With Eq. (\ref{eq.phi_fs}) we obtain $\partial \phi_f/\partial s =- k_i$
and
%its first order derivative as
\begin{equation} \label{eq.phif}
    \begin{aligned}
    \frac{\partial \phi_f}{\partial t} & = %\omega_{ce}(s_i(t)) - \frac{\partial k_i(t)}{\partial t}(s-s_i(t)) + k_i v_r
    %\\
     \omega_i - \frac{\partial k_i(t)}{\partial t}(s-s_i(t))
   % \\
    \simeq \omega_i~,
    \end{aligned}
\end{equation}
where the resonance condition Eq. (\ref{eq.resonance}) has been used. 
% and 
% %the spatial derivative is 
% \begin{equation}
%     \frac{\partial \phi_f}{\partial s} \simeq - k_i~.
% \end{equation}
%Then a linear operator may be defined for the left-hand-side of   
Then the wave equation (\ref{eq.wavemid_1}) becomes 
\begin{equation}\label{eq.wave}
 e^{\imath \phi_f} \hat{L} a(s_i,t)-\frac{4 \pi}{c}e^{\imath \phi_f}j_w(s_i,t)=\frac{4 \pi}{c} e^{\imath \phi_f}j_p(s_i,t),
\end{equation}
where  Eq. (\ref{eq.kernel_j}) has been used for the current envelope
and                
the linear operator $\hat{L} $ is defined as  
\begin{equation}
    \hat{L} a=  \frac{1}{c^2}\frac{\partial^2 a}{\partial t^2} - \frac{\partial^2 a}{\partial s^2} + \frac{2 \imath \omega_i}{c^2} \frac{\partial a}{\partial t}+ 2 \imath k_i \frac{\partial a}{\partial s} + (k_i^2 - \frac{\omega_i^2}{c^2})a.
\end{equation}


%For the right-hand-side of the wave equation, we first deal with the linear current.
For  the high frequency whistler wave, 
%the ion is too heavy to respond the wave perturbation, thus 
the linear current can be derived from the equation of motion of cold electrons \cite{stix1992},
\begin{equation}
    \frac{\partial J_w}{\partial t}-\imath \omega_{c e}(s) J_w=-\frac{\omega_{p}^{2}(s)}{4 \pi c} \frac{\partial A}{\partial t},
\end{equation}
where 
$\omega^2_{p} = 4 \pi n_{e} e^2 /m_e$
%$\omega_{p}$ 
is the plasma frequency of background electrons.
With the initial condition $J_w(t=0) = 0$ the  solution to this first-order inhomogeneous differential equation is
\begin{equation}
    J_w(t) = - \frac{\omega_{p}^2(s)}{4 \pi c}  e^{\imath \omega_{ce}(s)t} \int_0^t \frac{\partial A}{\partial \tau} e^{-\imath \omega_{ce}(s)\tau} \mathrm{d} \tau  ~.
\end{equation}
% the constant $\mathrm{C}$ vanishes in the general solution, and gives the linear current as,
% \begin{equation}
%     \label{eq.jw}
%     j_w =-\frac{\omega_{p}^{2}(s)}{4 \pi c} \int_0^{t} \mathrm{d} \tau \frac{\partial A}{\partial \tau} e^{\imath \omega_{c e}(s)(t-\tau)}~.
% \end{equation}
% Substituting  
% Eq. 
% %into the above equation, 
% we haveUsing  Eqs. (\ref{eq.kernel_A}) 
From Eq. (\ref{eq.phif}) we have 
\begin{equation}
    \phi_f(s,\tau) \simeq \phi_f(s,t) +  (\tau-t) \omega_{i}.
 %   \phi_f(s,\tau) =\phi_f(s,t) + \int_t^\tau \omega_{i}(k_i(\tau^\prime),\tau^\prime) \mathrm{d}\tau^\prime
\end{equation}
Thus 
 the envelope of the linear current is 
% \begin{equation}
%  {4 \pi \over c}   j_w = -\frac{\omega_p^2}{ c^2}\int_0^t \mathrm{d} \tau \left(\frac{\partial a}{\partial \tau} + \imath \omega_i a\right)e^{\imath \phi_f(s,\tau)} e^{\imath \omega_{ce}(t-\tau)}~,
% \end{equation}
\begin{equation}
 {4 \pi \over c}   j_w = -\frac{\omega_p^2}{ c^2}\int_0^t \mathrm{d} \tau \left(\frac{\partial a}{\partial \tau} + \imath \omega_i a\right)e^{\imath 
(\omega_i - \omega_{ce})(\tau-t)}.
\end{equation}
%phase term of the linear current, we write 
%The last term can be approximately written as
%As to , Similar to the wave eikonal, we can write the fast varying phase $\phi_f(s,\tau)$  approximately as, 
%and 
%\begin{equation}
% $   e^{\imath \omega_{ce}(t-\tau)} = e^{\imath \int^t_\tau \omega_{ce}\mathrm{d}\tau^\prime}$.
%\end{equation}
% \begin{equation}
%   \begin{aligned}
%   {4 \pi \over c} j_w &= -\frac{\omega_p^2}{ c^2} e^{\imath \phi_f(s,t)}
%   \int_0^t \mathrm{d} \tau \left(\frac{\partial a}{\partial \tau} + \imath \omega_i a\right) e^{\imath (\tau-t) (\omega_i - \omega_{ce})}
%   %e^{\imath \int_t^\tau (\omega_i - \omega_{ce})\mathrm{d}\tau^\prime}~.
%   \end{aligned}
% \end{equation}
%where we have used the approximation
%For the integral with the zeroth order derivative of $a$, we can 
With the initial condition  $a(t=0) = 0$ 
the integration by parts for the  integral 
yields
%  where
 %\begin{widetext}
\begin{equation}
    \begin{aligned}
    &\int_0^t \imath\omega_i a e^{\imath  (\omega_i - \omega_{ce})(\tau-t)}\mathrm{d} \tau   = \frac{\omega_i a}{\omega_i - \omega_{ce}}
    \\
    &-\int_0^t \frac{\omega_i}{\omega_i - \omega_{ce}}  e^{\imath (\omega_i - \omega_{ce})(\tau-t)} \frac{\partial a}{\partial \tau}  \mathrm{d} \tau ~.
    \end{aligned}
\end{equation}
% \end{widetext}
%\begin{widetext}
Substituting the  linear current envelope into the wave equation (\ref{eq.wave}) we have
\begin{equation}\label{eq.wave2}
    \begin{aligned}
    &  \hat{L} a+\frac{\omega_p^2 \omega_i}{c^2\left(\omega_i-\omega_{c e}\right)} a \\
    & +\frac{\omega_p^2}{c^2} \int_0^t d \tau \frac{\omega_{c e}}{\omega_{c e}-\omega_i} \frac{\partial a}{\partial \tau} e^{-\imath (t-\tau)\left(\omega_i-\omega_{c e}\right) } =\frac{4\pi}{c} j_{p}(s_i,t).
    \end{aligned}
\end{equation}
%\end{widetext}
Note that the fast varying phase term $e^{\imath \phi_f}$ has been eliminated from both sides of the wave equation. 
%Note that the linear current term has been analytic merged on the left-hand-side, and the right only contains 


The current term from energetic particle 
is obtained through the integration  of the perturbed distribution functions over the phase space in the vicinity of the resonances, 
\begin{equation}\label{eq.nonlinear_Jp}
    J_p(s, t)=-e n_{h 0} \iiint f\left(s, p_{\parallel}, \mu, \varphi ; t\right) \sqrt{\frac{2 \omega_{ce} \mu}{m_{e}}} e^{\imath \varphi} d p_{\parallel} d \mu d \varphi~,
\end{equation}
Note that $d p_{\parallel} d \mu d \varphi=k_i d \xi \rm d \Omega \rm d \mathcal{J}$
where the determinant of the Jacobian matrix  is calculated from Eqs. (\ref{eq.newQ}) and (\ref{eq.oldP}). 
% \begin{equation}
%     \frac{\partial (p_\|,\mu,\varphi)}{\partial (\xi,\Omega,\mathcal{J})} = k~.
% \end{equation}
% Similar to the slowly varying kernel in Eq. (\ref{eq.kernel_A})
% the plasma current 
% \begin{equation}
% j_{p}(s,t)\simeq e^{\imath \phi_f}  j_p(s_i, t)~,
% %j_{p}(s_i,t) = \exp(\imath(\xi-\varphi))j_{p}(s,t)~.
% \end{equation}
% where  $j_{p}(s_i,t)$ is the current envelope on the cell.  
%Since $\phi_f=\varphi-\xi$and $\mu=\mathcal{J}+\Omega+\Pi_i$, 
 %the  current term 
Using Eqs. (\ref{eq.nonlinear_Jp}) 
and 
(\ref{eq.kernel_j})
we obtain 
the envelope of energetic particle current  on the cell,
\begin{equation}\label{eq.nonlinear_J}
\begin{aligned}
 {4 \pi \over c}    j_p(s_i,t) &=
 - \frac{\omega_{h0}^2k_i}{ec}\iiint \sqrt{2m_e\omega_{ce}(s)(\mathcal{J}+\Omega+\Pi_i)}\\
 &\times f(\xi,\Omega,\mathcal{J};s_i(t),t)e^{\imath \xi} \rm d \xi \rm d \Omega \rm d \mathcal{J}~,   
    \end{aligned}
\end{equation}
where $\omega^2_{h0} = 4 \pi n_{h0} e^2 /m_e$ is the plasma frequency of the energetic electrons.

To examine the onset of  whistler-mode chorus, we use the whistler dispersion relation,
\begin{equation}
k_i^2 - \frac{\omega_i^2}{c^2}+\frac{\omega_p^2 \omega_i}{c^2\left(\omega_i-\omega_{c e}\right)}=0.
\end{equation}
%we set the wave frequency and wave number to be those of  
%Using the onset condition, i.e., set the frame center at
% the most unstable mode driven by energetic electrons: 
%$\omega_i=\omega_l$ 
%and 
%$k_i=k_l$. 
%we have
%For the whistler-mode chorus in the resonance frame, 
Then the second-order wave envelope  equation  (\ref{eq.wave2}) in the resonance frame is  
%\begin{widetext}
\begin{equation}\label{eq.wave_2nd}
    \begin{aligned}
   & \frac{1}{c^2}\frac{\partial^2 a}{\partial t^2} - \frac{\partial^2 a}{\partial s_i^2} + \frac{2\imath\omega_i}{c^2}\frac{\partial a}{\partial t} + 2\imath k_i\frac{\partial a}{\partial s_i} \\
   &+ \frac{\omega_p^2 \omega_{c e}}{c^2(\omega_{c e}-\omega_i)} \int_0^t d \tau \frac{\partial a}{\partial \tau} e^{-\imath\left(\omega_i-\omega_{c e}\right)(t-\tau)} = 
    {4 \pi \over c}    j_p(s_i,t) .
    %- \frac{\omega_{h0}^2k_l}{ec}\iiint \sqrt{2m_e\omega_{ce}(s)(\mathcal{J}+\Omega+\Pi_i(t))}f(\xi,\Omega,\mathcal{J};s_i(t),t)e^{\imath \xi} \rm d \xi \rm d \Omega \rm d \mathcal{J}~.    
    \end{aligned}
\end{equation}
%\end{widetext}





Finally, we  derive a first-order wave equation for the onset of chorus. 
For the slowly varying envelope, the second-order time and spatial derivatives for the amplitude are generally small compared to the first-order derivatives. 
Thus we further neglect the second-order terms in the wave envelope equation
(\ref{eq.wave_2nd})
and simplify 
the integral term %in the equation
% can be further simplified 
using integration by parts,
%\begin{widetext}
\begin{equation}
\begin{aligned}
\int_0^{t} d \tau \frac{\partial a}{\partial \tau} e^{-\imath\left(\omega_{i}-\omega_{c e}\right)(t-\tau)}
    %&=\int^{t}\frac{\partial a}{\partial\tau} \frac{1}{\imath\left(\omega_l-\omega_{ce}\right)}d e^{-\imath\left(\omega_l-\omega_{ce}\right)\left(t-\tau\right)}
%\\
%    &=\left.\frac{\partial a}{\partial \tau}\frac{e^{-\imath\left(\omega_l-\omega_{ce}\right)\left(t-\tau\right)}}{\imath\left(\omega_l-\omega_{ce}\right)}\right|^{t}_{0} - \int^{t} \frac{e^{-\imath\left(\omega_l-\omega_{ce}\right)\left(t-\tau\right)}}{\imath\left(\omega_l-\omega_{ce}\right)}d \frac{\partial a}{\partial\tau}
   % \\
    \simeq \frac{\partial a}{\partial t}\frac{1}{\imath\left(\omega_{i}-\omega_{ce}\right)}~,
\end{aligned}
\end{equation}
%\end{widetext}
%Note that we also neglect the integral term that contains 
where the second-order  derivative of $a$ has been neglected
and  the initial condition  $\partial a/\partial t = 0$ at $t=0$ has been applied.
%After some simple algebra, we reduce 
Then Eq. (\ref{eq.wave_2nd}) is reduced to the first-order wave envelope equation,
\begin{equation}\label{eq.wave_1st}
    \frac{\partial a}{\partial t} +v_g \frac{\partial a}{\partial s_i} =\frac{-\imath 2\pi v_g}{c k_i } j_p,
\end{equation}
where  $v_g = 2 k_i c^2 / (2\omega_i + \omega_p^2 \omega_{ce}/(\omega_{ce}-\omega_i)^2) $  is the group velocity of the linear whistler wave.
Using Eq. (\ref{eq.wave_1st}) we obtain the evolution of the wave amplitude,
\begin{equation}
    \frac{\partial |a|^2}{\partial t} +v_g \frac{\partial |a|^2}{\partial s_i} =\frac{ 4\pi v_g}{  ck_i} Re( -\imath j_p a^*),
\end{equation}
where $|a|^2=a^* a$ represents the wave energy and $Re( -\imath j_p a^*)$ determines the wave-particle power transfer. 

