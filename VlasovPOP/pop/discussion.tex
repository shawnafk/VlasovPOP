\section{Discussion}
\label{sec:dis}
\subsection{The onset stage of frequency chirping}
Recalling cell definition, it is a local reference frame that tracks the local resonant particle at rest.
However, when the chirping begins, it is hard to follow the resonance particle precisely since the local resonance changes all the time.
To reduce the complexity of the dynamics system without losing any generality, we here consider the initial time stage of the chirping problem, i.e., the onset problem.
The variation of wave frequency is considerably small in the onset stage of the instability. 
Thus, the resonance frame is approximately centered at the most unstable frequency $\omega_l$ and the corresponding wave number $k_l$ obtained from the linear wave dispersion.
By simply replacing $k_i$, $\omega_i$ by $\omega_l$ and $k_l$ in the related parameters and equations, we can greatly reduce the motion equation.
For the perturbed motion, the parameter $alpha$ is greatly reduced. Since the most unstable frequency is a temporospatial constant, $\partial \omega/\partial t$ is vanished in Eq.~(\ref{eq.alp1}).
While the exact time derivative of $k_l$ still has dependence on $s$, i.e.,
\begin{equation}
    \frac{\mathrm{d}k_l}{\mathrm{d}t}  = v_r \frac{\partial k_l}{\partial s}~.
\end{equation}
According to the linear dispersion
\begin{equation}
    c^2 k_l^2=\omega_l^2+\frac{\omega_{pe}^2\omega_l }{\omega_{ce}-\omega_l}~,
\end{equation}
do differentiate on both sides with respect to $s$ yields
\begin{equation}
    \begin{aligned}
    2 c^2 k_l \frac{\partial k_l}{\partial s}&=\left(2 \omega_l+\frac{\omega_{ce} \omega_{pe}^2}{\left(\omega_{ce}-\omega_l\right)^2}\right) \frac{\partial \omega_l}{\partial s}-\frac{\omega_l \omega_{pe}^2}{\left(\omega_{ce}-\omega_l\right)^2} \frac{\partial \omega_{ce}}{\partial s}
    \\
    & = -\frac{\omega_l \omega_{pe}^2}{\left(\omega_{ce}-\omega_l\right)^2} \frac{\partial \omega_{ce}}{\partial s}~,
    \end{aligned}
\end{equation}
i.e.,
\begin{equation}
    \frac{\partial k_l}{\partial s} = -\frac{\omega_l \omega_{pe}^2}{2 c^2 k_l \left(\omega_{ce}-\omega_l\right)^2} \frac{\partial \omega_{ce}}{\partial s}~.
\end{equation}
Note that we neglect the dependence of $\omega_{pe}$ on $s$.
\begin{equation}
    \begin{aligned}
        \frac{\partial k_l}{\partial s} &=- \frac{c^2 k_l^2 - \omega_l^2}{2 c^2 k_l^2}\frac{k_l}{(\omega_{ce}-\omega_l)} \frac{\partial \omega_{ce}}{\partial s}
        \\
        &=-\frac{1}{v_r} (\frac{1}{2} - \frac{\omega_l^2}{2c^2 k_l^2}) \frac{\partial \omega_{ce}}{\partial s}
        \\
        &\simeq \frac{1}{2v_r}\frac{\partial \omega_{ce}}{\partial s}~.
    \end{aligned}
\end{equation}
The phase velocity is close to the speed of light for the whistler dispersion at the frequency of interest, thus we have the above approximation.

The onset condition finally gives 
\begin{equation}\label{eq.alp1}
   \alpha \simeq \frac{k_l}{\omega_b^2}\left(\mathcal{J}-\frac{\Pi_i}{2}\right) \frac{\mathrm{d} \omega_{c e}}{\mathrm{~d} s}
    \end{equation}

In addition, with the onset condition, the motion equation (\ref{eq.m3}) for $\mathcal{J}$ reduces to
\begin{equation}
        \frac{\mathrm{d}\mathcal{J}}{\mathrm{d}t} \simeq  \frac{m_e(\omega_l - \omega_{ce})}{k_l^3} \frac{\mathrm{d} k_l}{\mathrm{d} t} + \frac{m_e(\omega_l - \omega_{ce})}{k_l^3} \frac{\partial \omega_{ce}}{\partial s_i}  +\frac{\mathcal{J}}{k_l}\frac{\partial \omega_{ce}}{\partial s_i}~.
\end{equation}
Multiplying $\omega_l - \omega_{ce}$ on both sides to construct $v_r$ on the spatial derivative of $\omega_{ce}$ that change the spatial derivative to exact time derivative, i.e., 
\begin{equation}
    (\omega_l - \omega_{ce}) \frac{\mathrm{d}\mathcal{J}}{\mathrm{d}t} \simeq  \frac{m_e(\omega_l - \omega_{ce})^2}{k_l^3} \frac{\mathrm{d} k_l}{\mathrm{d} s_i} + \frac{m_e(\omega_l - \omega_{ce})}{k_l^2} \frac{\mathrm{d} \omega_{ce}}{\mathrm{d} t} + \mathcal{J}\frac{\mathrm{d} \omega_{ce}}{\mathrm{d} t}~.
\end{equation}
Further using the identity $\mathrm{d}\omega_l/\mathrm{d}t \equiv 0$, we obtain the following relation
\begin{equation}
    (\omega_l - \omega_{ce}) \frac{\mathrm{d}\mathcal{J}}{\mathrm{d}t} + \mathcal{J}\frac{\mathrm{d}}{\mathrm{d} t}(\omega_l - \omega_{ce}) \simeq - m_e\left((\omega_l - \omega_{ce})^2 \frac{\mathrm{d}}{\mathrm{d} t}\frac{1}{2 k_l^2} + \frac{1}{2 k_l^2} \frac{\mathrm{d}}{\mathrm{d} t}(\omega_l - \omega_{ce})^2\right)~.
\end{equation}
The equation is clearly to be analytically integrated, and yields
\begin{equation}\label{eq.Jcons}
    (\omega_l - \omega_{ce})\mathcal{J} +  \frac{m_e(\omega_l - \omega_{ce})^2}{2k_l^2} = \mathrm{Const.}
\end{equation}
The constant for each cell is determined by the initial choice of $\mathcal{J}$, which give the entire information of the dynamics on the slowly varying scale along the magnetic field line.

\subsection{The adiabatic theory}
%solve current integral and obtain the chirping law
In the previous section, we have shown the nonlinear current as the velocity momentum of perturbed distribution function. 
The current dominates the nonlinear behavior of the chorus chirping process, and under certain condition, we can simplify the integral and show qualitatively how does the frequency chirp.
The distribution function of the trapped electrons forms a hole in the phase space at the nonlinear stage, and we consider the deviation from the unperturbed distribution $\Delta f$, i.e., the depth of the hole.
The nonlinear current is directly determined by $\Delta f$, since the equilibrium distribution does not contribute the current.

%the adiabatic invariant 
For those electrons trapped by the slowly varying wave envelope and circling around in the phase space, we can define the adiabatic invariant at a given $s_i,\mathcal{J}$,
\begin{equation}\label{eq.def_I}
    \mathcal{I} = \frac{1}{2\pi} \oint \Omega(H_i,\xi,t) \mathrm{d} \xi~,
\end{equation}
where the momentum $\Omega$ is the function of local Hamiltonian.
%and the distribution expansion
The deviation can be written as the function of the adiabatic invariant, i.e., $\Delta f(s_i,\mathcal{J},\mathcal{I},\xi,t)$.

We replace $f$ by $\Delta f$ in the current integral in Eq. (\ref{eq.nonlinear_J}) and directly change the integral to 
\begin{equation}
    j_p(s_i,t) \approx - \sqrt{2m_e\omega_{ce}(s)(\mathcal{J} + \Pi_i(t))}\int\mathrm{d} \mathcal{J} \int_0^{I_{\mathrm{s p x}}}  \mathrm{d}\mathcal{I}  \int_0^{2\pi} \mathrm{d}\psi  \Delta f(s_i,\mathcal{J},\mathcal{I},\xi,t)e^{\imath \xi}  ~,
\end{equation}
since $\mathrm{d}\xi\mathrm{d}\Omega = \mathrm{d}\mathcal{I}\mathrm{d}\psi$, where $\psi$ is the angle variable of $\mathcal{I}$.
From the Jacobi of the differential element, the integral over $\psi$ 
\begin{equation}
      \int f \mathrm{d}\psi = \int f \frac{\mathrm{d}\Omega}{\mathrm{d}\mathcal{I}}\mathrm{d}\xi = \frac{\partial H}{\partial \mathcal{I}}  \int f \frac{\mathrm{d}\Omega}{\mathrm{d} H}\mathrm{d}\xi = \langle f \rangle
\end{equation}
where $\langle ... \rangle$ denotes the bounce average as given in ref. \cite{berk1999}.
Thus the current integral becomes
\begin{equation}
    j_p(s_i,t) \approx -  {2\pi \tau} \sqrt{2m_e\omega_{ce}(s)(\mathcal{J} + \Pi_i(t))}\int\mathrm{d} \mathcal{J} \int_0^{I_{\mathrm{s p x}}}\mathrm{d}\mathcal{I}  \langle \Delta f(s_i,\mathcal{J},\mathcal{I},\xi,t)e^{\imath \xi} \rangle  ~.
\end{equation}
For the slowly varying wave field, $\Delta f$ can be expanded in powers of small scale $\epsilon$ satisfies condition \cite{berk1999}
\begin{equation}
    %\left[\left|\frac{\mathrm{d} \omega_b}{\mathrm{~d} t}\right|,\left|\frac{\mathrm{d} \omega}{\mathrm{d} t}\right|\right] \ll \omega_b^2,
    \epsilon \equiv \mathrm{s p x}\left(\ddot{\delta \omega}/\omega_b^3, \dot{\delta \omega}/\omega_b^2, \omega_b/\delta \omega \right) \ll 1~,
\end{equation}
and follows the derivation in ref. \cite{berk1999}, we keep to the first order term, the average integrals are
\begin{equation}
    \begin{aligned}
    \langle\Delta f \sin \xi \rangle &\simeq \alpha \Delta f_0 ~, \\ 
    \langle \Delta f \cos \xi \rangle &\simeq  \Delta f_0 \langle \cos \xi \rangle ~.
    \end{aligned}
\end{equation}
%approximation 2, I is constant over trapped region
Moreover, we further assume that $\Delta f_0$ is independent with $\mathcal{I}$, which indicates the depth of the hole is flat with in the enclosed phase space region, i.e., the water bag approximation \cite{omura_theory_2008,hezaveh2021}. Therefore, the integral over $\mathcal{I}$ only depend on the $I_\mathcal{max}$ which is the one on the separatrix. 
According to equation (\ref{eq.def_I}), the integral becomes
\begin{equation}
    \int^{\mathcal{I}_{s p x}}_0 \mathrm{d}\mathcal{I} = \mathcal{I}_{s p x} \equiv \oint_{s p x} \Omega (\xi) d \xi.
\end{equation}
where the boundary trapped particle phase space hole can be analytically given,
\begin{equation}
    \Omega(\xi) = \pm \frac{\omega_b}{k^2} \sqrt{2 (e_{spx}-\cos \xi - \alpha \xi)}~,
\end{equation}
$e_{spx}$ is the Hamiltonian on the separatrix.

Here we define two functions of $\alpha$
\begin{equation}
    \begin{aligned}
        m_{spx}(\alpha) & \equiv \langle \cos \xi \rangle  \mathcal{I}_{spx} = \frac{\sqrt{2} \omega_b}{\kappa^2} \oint_{s p x} d \xi \cos \xi \sqrt{e_{s p x}-\cos \xi-\alpha \xi} \\
        n_{spx}(\alpha) & \equiv \alpha \mathcal{I}_{spx} = \alpha \frac{\sqrt{2} \omega_b}{\kappa^2} \oint_{s p x} d \xi \sqrt{e_{s p x}-\cos \xi-\alpha \xi} ~.
    \end{aligned}
\end{equation}
The current can be directly obtained as 
\begin{equation}\label{eq.adi_J}
    j_p \approx \frac{2 \omega_b}{\kappa^2} \sqrt{\omega_{c e}(s)(J+\Pi)}\left(m_{s p x}+\imath ~ n_{s p x}\right) \int d \mathcal{J} \Delta f(\mathcal{J},s_i,t) ~.
\end{equation}
%the nonlinaer currnet and the chirping rate ...
%differnece with omura 
%process approxiamtion strict follows adiabatic approximation
%why the hole is oblique
%alpha value should small to validate approximation

%figure