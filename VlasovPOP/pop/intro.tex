\section{Introduction}
\label{sec:intro}
Whistler-mode chorus is a special electromagnetic emission in whistler range of frequency which are frequently observed in the planetary magnetosphere \cite{burtis_magnetospheric_1976,tsurutani_postmidnight_1974}.
The chorus is related with the energy transference between wave and energetic electrons, thus plays a key role in particle acceleration processes in the radiation belt \cite{horne_wave_2005,thorne_rapid_2013,reeves_electron_2013} and pulsating and diffuse aurora in the atmosphere \cite{nishimura_identifying_2010,kasahara_pulsating_2018,thorne_scattering_2010}. Thus it has received a significant research interest.  

The physics of the chorus wave in essence are the nonlinear wave-particle interactions between the resonant trapped electrons and the whistler waves. 
The similar nonlinear wave-particle interactions, such as the Alfv\'en wave instabilities \cite{chen2016,wang2018}, have been studies extensively in the fusion related plasma.
Such instabilities also involve the mode frequency sweeping and lead to premature ejection of alpha particles that deteriorate plasma confinement \cite{fasoli2007}.
The dynamics of nonlinear resonant particle can be modeled by Berk-Brizman (BB) model \cite{berk_spontaneous_1997}, which is a general theoretic model based on the bump-on-tail (BOT) paradigm. 
The original version and the advanced versions \cite{lilley_formation_2014} focus on the uniform regime, where the wave is treated as stationary with fixed wave number $k$. 
The spontaneous hole and clump structure and their evolution in the phase space were revealed and well explains the frequency chirping of the excited wave.
For the chorus wave case, the phase space behavior of the resonant electrons also plays an important role for the  chirping behavior. 
It has been long time discussed \cite{sudan_theory_1971} and quantitatively explains the frequency chirping rate and nonlinear wave growth of chorus wave at various locations along the magnetic field \cite{omura_theory_2008,omura_nonlinear_2021}.

However, unlike the BB model for the BOT paradigm, a key feature for the chorus wave in the planetary magnetosphere is the involved inhomogeneous magnetic field. 
It was first noticed by Helliwell in his kinematic theory of the ``consistent-wave'' concept in the 1960s \cite{helliwell_theory_1967}.
It elucidates the change of chirping rate with respect to magnetic field inhomogeneity.
In more recent studies \cite{wu_controlling_2020,fujiwara2023}, it is found that the frequency chirping behavior can be controlled by changing the magnetic field configuration. Both downward chirping and bi-direction chirping were reproduced.
And a phenomenological chirping model called Trap-Release-Amplify (TaRA) model was proposed and explained how frequency chirping occurs and why chirping direction with respect to the sign of magnetic field gradient.
Although the model has successfully explained chirping behavior in various planetary magnetosphere, including that on Mars \cite{teng2023}, it raises the intriguing question of how rapidly varying chirping elements and resonant electrons are influenced by the background magnetic field's inhomogeneity. This is despite the fact that the scale of the field's inhomogeneity is significantly larger than the characteristic scales of rapid changes and fine structure.
Besides, numerical simulations of artificially triggered emissions show different mechanism for the formation of chorus wave \cite{nogi2022,nogi2023}. 
The wave source moves with a velocity that is a combination of the group velocity and resonance velocity, rather than the velocity alone. This suggests that the chirping may not be emitted by the released particle.

Anyhow, in current theoretical and numerical studies, many questions still remain regarding the fine structures of the chorus observed in the magnetosphere. \cite{zhang2021}. 
The chirping mechanism and nonlinear wave particle interaction questions are also undercover as pointed by Tao, et al. in the recently studies \cite{tao_theoretical_2020,tao_trap-release-amplify_2021}. Additional information about the particle dynamics and wave evolution such as high resolution phase-space structure, fine wave number and frequency results, are still desired under current models.
However, due to introduce background magnetic field inhomogeneity, the chorus is essentially a multiple problem. 
Finding an appropriate method to separate the scales of resonant particles is a complex task because of the intricate interactions between waves and particles along the inhomogeneous magnetic field. 
Challenges and the need have spurred the development of a new theoretical and numerical model.

In this paper, we go over the dynamics of resonant electrons and the evolution of whistler wave in Earth inner magnetosphere. The key feature in our theory is the scale separation of electron motion to the slowly one along the magnetic field and the fast wave interaction. 
To do so, we build a Hamiltonian theory in the reference frame moving with local resonance.
We divide the continuous spatial domain of the single magnetic field line into a group of cells representing a group of local reference.
By expanding the phase of the wave and the resonance electrons, a canonical Hamiltonian on the cell is constructed which naturally separated the fast and slowly varying motion of resonant electrons.
The canonical motion equations are then derived, which governs the dynamics of resonant electrons. 
The evolution of whistler-mode wave is determined by the Ampere equation, in which the wave is expressed in its eikonal form, thus, the frequency and wave number can be given directly and accurately.

We organize our paper as follows. In section~\ref{sec:theory}, we present the Hamiltonian theory and derive the Vlasov equation for the resonant particle distribution, and the wave equation for the slowly varying wave envelope.
In section~\ref{sec:dis}, we discuss the onset stage of the chirping, which gives a more brief form for the local Hamiltonian, and derives an integral of motion for the resonant electrons interact with the chirping wave in an inhomogeneous magnetic field. We also discuss the adiabatic regime for the resonant electron phase space behavior, and give the chirping rate at typical parameters.  
Finally, the summary is presented in section~\ref{sec:conc}.
% MOVE to CPC
%Based on the equations of particles and waves, we developed a novel Vlasov-Ampere simulation model. The evolution of particle distribution function is directly solved, and combined with the wave equation, the generation and propagation of chorus wave can be given self-consistently. In this paper, using this numerical model, we first provide the generation process of the rising-tone chorus, where the wave araises from linear whistler wave growth and begins chirping after enters the nonlinear stage, in which growth rate and frequency chirping rate can be given precisely. The role of in-phase and out-phase nonlinear current that plays during the nonlinear growth of the chorus wave are investigated. The results are in fully consistent with previous studies, and additionally provide some unique insight on the necessity of a time dependent wave number during the nonlinear stage. Finally, we presented the detailed phase space structure and its evolution, which indicated the generation mechanism of chorus wave. 
%MOVE to CPC After introduce the theory, we can now go to the numerical schemes 
%Besides various theoretical works, numerical models and simulations are also indispensable in the studies of the whistler-mode chorus. To understand the complicated nonlinear dynamics of particles as well as the evolution of waves, various numerical have been developed, such as the reduced Vlasov Hybrid Simulation (VHS) code~\cite{nunn_numerical_1990}, and the particle-in-cell (PIC) methods, including the full particle-in-cell (PIC) code\cite{hikishima_full_2009} and some electron hybrid code~\cite{katoh_computer_2007,tao_numerical_2014}; Those simulations provided many supportive and informative results. The generation of chorus wave, the chirping rate, and wave growth etc. have been successfully presented in simulation. 
